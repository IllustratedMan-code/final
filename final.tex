% Created 2022-11-28 Mon 13:06
% Intended LaTeX compiler: lualatex
\documentclass[11pt]{article}
\usepackage{graphicx}
\usepackage{longtable}
\usepackage{wrapfig}
\usepackage{rotating}
\usepackage[normalem]{ulem}
\usepackage{amsmath}
\usepackage{amssymb}
\usepackage{capt-of}
\usepackage{hyperref}
\usepackage{minted}
\author{David Lewis}
\date{\today}
\title{Introduction to \LaTeX{}}
\hypersetup{
 pdfauthor={David Lewis},
 pdftitle={Introduction to \LaTeX{}},
 pdfkeywords={},
 pdfsubject={},
 pdfcreator={Emacs 28.2 (Org mode 9.6)}, 
 pdflang={English}}
\makeatletter
\newcommand{\citeprocitem}[2]{\hyper@linkstart{cite}{citeproc_bib_item_#1}#2\hyper@linkend}
\makeatother

\usepackage[notquote]{hanging}
\begin{document}

\maketitle
\tableofcontents


\section*{Introduction}
\label{sec:org22d776d}
\LaTeX{} is a ``document preparation system'' (\citeprocitem{8}{\textit{LaTeX - a Document Preparation System}, n.d.}). It
is a program that takes a \emph{plain text}\footnote{``Plain text'' refers to text without formatting. Font, margin, and spacing information are all absent. Files containing plain text often end with the \texttt{.txt} extension.} file written in a special
syntax and creates a document, commonly in the PDF format. Rather than display
the formatting in the editor, a la Microsoft Word (\citeprocitem{11}{\textit{Microsoft Word – Word Processing Software | Microsoft 365}, n.d.}), \LaTeX{} uses a set of codes or
``commands'' to determine how the final document will look. Italicized text for
example, would look \texttt{\textbackslash{}emph\{like this\}} in the text file, but \emph{like this} in the
final document. This document aims to provide a brief introduction in the use of
\LaTeX{}, covering everything from general syntax to document classes.
\subsection*{Prerequisite Knowledge}
\label{sec:org553299c}
The following skills are required to read this document successfully:
\begin{itemize}
\item English literacy, there will be some jargon
\item Computer skills, basic terminology\footnote{Basic terminology includes terms like ``window'', ``software'', ``webpage''. More
complicated terms will either be defined, or an appropriate \href{https://en.wikipedia.org/wiki/Hyperlink}{hyperlink} will be used.}
\item Ability to follow detailed instructions
\item Ability to use a \emph{text editor}\footnote{A \href{https://en.wikipedia.org/wiki/Text\_editor}{text editor} is a piece of software that edits plain text. See
the extended definition \hyperref[sec:org5c3d942]{here}.}, instructions will be given using
\href{https://code.visualstudio.com/}{Visual Studio Code} (VScode)(\citeprocitem{15}{\textit{Visual Studio Code - Code Editing. Redefined}, n.d.--a})
\item If using Linux, basic terminal knowledge is assumed\footnote{Basic terminal knowledge includes moving directories(\texttt{cd} command in Unix operating systems) and running commands.}
\end{itemize}
\subsection*{Uses}
\label{sec:orgfda4981}
In the case of Big Publishing CO., we use it for publishing. Authors submit
their work in an arbitrary format, which might include handwritten equations,
non-standardized fonts, etc. \LaTeX{} provides a way to automate and standardize
the formatting of these documents. It is also frequently used by document
creators to export to multiple formats without extra work.

\section*{A Basic Example}
\label{sec:org278c78b}
On first glance, \LaTeX{} looks a bit like a [programming language]. For the most
part, \LaTeX{} would \textbf{not} be considered a programming language. It is more like
HTML (\citeprocitem{7}{\textit{HTML Standard}, n.d.}) or MediaWiki markup (\citeprocitem{6}{\textit{Help}, n.d.}).
These languages are generally considered \href{https://en.wikipedia.org/wiki/Markup\_language}{Markup languages}. Markup languages are
much easier to learn in comparison to programming languages. Here is a very short
example document in \LaTeX{}:
\begin{minted}[fontsize=\scriptsize]{latex}
\documentclass{article}
\begin{document}
Hello, World! I just wrote my first \LaTeX{} document!
\end{document}
\end{minted}
\begin{center}
\includegraphics[height=1cm]{hello_world.png}
\end{center}

This is the standard format for code in this document. There will always be
output after code blocks. \LaTeX{} uses the \texttt{\textbackslash{}} character to demarcate \emph{Commands}\footnote{Commands can be pretty advanced and are why \LaTeX{} can
sometimes be considered a programming language. Users don't create them very
often.} from the
rest of the text. \texttt{\{\}} characters are used to provide arguments to these commands.

\section*{Getting Started}
\label{sec:orgdd9d779}
There are many different ways to use \LaTeX{}. \LaTeX{} is a program, but it is also a
language. \LaTeX{} has been at least partially implemented in many different pieces of
software, particularly the math typesetting functionality, which will be
covered \hyperref[sec:orgdc497cc]{later}.

\subsection*{Installation}
\label{sec:org465be4b}
In this section, we will cover an overview of the installation of
the \LaTeX{} \emph{program} for every major platform.
\subsubsection*{Overleaf}
\label{sec:org459c9ad}
One of the easiest methods to get
started is to use \href{https://www.overleaf.com/}{Overleaf} (\citeprocitem{12}{\textit{Overleaf, Online LaTeX Editor}, n.d.}). Overleaf is an online
system that removes much of the hassle of creating documents in \LaTeX{}.
Overleaf's online naturre For
professional use, Overleaf is not recommended as data privacy is not assured.
This is a problem with all external web based tools. It
is great as a learning tool and for personal use however.
\subsubsection*{Windows}
\label{sec:org3e7badc}
The official recommended  way to install \LaTeX{} in Windows is to use the MiKTeX
(NO\_ITEM\_DATA:MikTex) distribution. Follow the \href{https://miktex.org/howto/install-miktex}{official instructions} to install. Make
sure the correct \href{https://miktex.org/kb/prerequisites-2-9}{version of Windows} is installed. Once installed, the program
\href{https://strawberryperl.com/}{Straberry Perl} (\citeprocitem{13}{\textit{Strawberry Perl for Windows}, n.d.}) must be installed. Assuming that
VScode is installed, the \href{https://marketplace.visualstudio.com/items?itemName=James-Yu.latex-workshop}{\LaTeX{} Workshop} extension must be installed through the
built in marketplace. \LaTeX{} can now be run through VScode if so desired.
\subsubsection*{MacOS}
\label{sec:org2bd838d}
The official recommended way to install \LaTeX{} in MacOS is to use the \href{https://tug.org/mactex/index.html}{MacTeX}
distribution(\citeprocitem{10}{\textit{MacTeX - TeX Users Group}, n.d.}). Follow the
\href{https://tug.org/mactex/mactex-download.html}{official instructions} to install. Assuming that VScode is installed, the \href{https://marketplace.visualstudio.com/items?itemName=James-Yu.latex-workshop}{\LaTeX{}
Workshop} extension must be installed through the built in marketplace. \LaTeX{}
can now be run through VScode if so desired.
\subsubsection*{Linux (Ubuntu)}
\label{sec:org86efe4b}
For Ubuntu linux, simply running:
\begin{minted}[fontsize=\scriptsize]{bash}
sudo apt update
sudo apt install texlive-full
sudo apt install latexmk
\end{minted}
will install the required packages. Assuming that VScode is installed, the \href{https://marketplace.visualstudio.com/items?itemName=James-Yu.latex-workshop}{\LaTeX{}
Workshop} extension must be installed through the built in marketplace. \LaTeX{}
can now be run through VScode if so desired.
\subsection*{First Document}
\label{sec:org1f3c40c}
Copy the code from \hyperref[sec:org278c78b]{above} into a file called `mydocument.tex` using a text editor.
If using VScode, \texttt{ctrl} + \texttt{alt} + \texttt{b} or \texttt{Build LaTeX} in the command palette should
build the document so that it looks like the output above.

If not using VScode, some terminal knowledge is required. Open a terminal in the
directory where \texttt{mydocument.tex} is located and run:
\begin{minted}[fontsize=\scriptsize]{bash}
latexmk mydocument.tex
\end{minted}
A document called \texttt{mydocument.pdf} should now appear in the same directory as \texttt{mydocument.tex}.

\href{https://mirrors.rit.edu/CTAN/support/latexmk/latexmk.pdf}{Latexmk} is a special command that automates much of the normal build process
(\citeprocitem{2}{\textit{CTAN: Package Latexmk}, n.d.}).
\section*{Basic Syntax}
\label{sec:org7ae4014}

\section*{Math Equations}
\label{sec:orgdc497cc}

\LaTeX{} is particularly useful for creating math equations.Many other pieces of
software have implemented an editor just for the math functionality of
\LaTeX{}.Examples include Microsoft Word, \href{https://www.libreoffice.org/}{Libreoffice}(\citeprocitem{9}{\textit{LibreOffice - Free Office Suite}, n.d.}) and even an \href{https://workspace.google.com/marketplace/app/autolatex\_equations/850293439076}{addon} for the
Google office suite(\citeprocitem{1}{\textit{Auto-LaTeX Equations - Google Workspace Marketplace}, n.d.}).

\subsection*{Examples}
\label{sec:org5339259}


\section*{References}
\label{sec:orge81e661}
\begin{hangparas}{1.5em}{1}
\hypertarget{citeproc_bib_item_1}{\textit{Auto-LaTeX Equations - Google Workspace Marketplace}. (n.d.). Retrieved November 28, 2022, from \url{https://workspace.google.com/marketplace/app/autolatex_equations/850293439076}}

\hypertarget{citeproc_bib_item_2}{\textit{CTAN: Package latexmk}. (n.d.). Retrieved November 28, 2022, from \url{https://ctan.org/pkg/latexmk/?lang=en}}

\hypertarget{citeproc_bib_item_3}{Foundation, B. (n.d.). blender.org - Home of the Blender project - Free and Open 3D Creation Software. \textit{Blender.Org}. Retrieved November 11, 2022, from \url{https://www.blender.org/}}

\hypertarget{citeproc_bib_item_4}{\textit{GitHub Copilot · Your AI pair programmer}. (n.d.). GitHub. Retrieved November 11, 2022, from \url{https://github.com/features/copilot}}

\hypertarget{citeproc_bib_item_5}{\textit{GNU Emacs - GNU Project}. (n.d.). Retrieved November 11, 2022, from \url{https://www.gnu.org/software/emacs/}}

\hypertarget{citeproc_bib_item_6}{\textit{Help:Formatting - MediaWiki}. (n.d.). Retrieved November 28, 2022, from \url{https://www.mediawiki.org/wiki/Help:Formatting}}

\hypertarget{citeproc_bib_item_7}{\textit{HTML Standard}. (n.d.). Retrieved November 28, 2022, from \url{https://html.spec.whatwg.org/}}

\hypertarget{citeproc_bib_item_8}{\textit{LaTeX - A document preparation system}. (n.d.). Retrieved November 11, 2022, from \url{https://www.latex-project.org/}}

\hypertarget{citeproc_bib_item_9}{\textit{LibreOffice - Free Office Suite}. (n.d.). Retrieved November 28, 2022, from \url{https://www.libreoffice.org/}}

\hypertarget{citeproc_bib_item_10}{\textit{MacTeX - TeX Users Group}. (n.d.). Retrieved November 28, 2022, from \url{https://tug.org/mactex/index.html}}

\hypertarget{citeproc_bib_item_11}{\textit{Microsoft Word – Word Processing Software | Microsoft 365}. (n.d.). Retrieved November 28, 2022, from \url{https://www.microsoft.com/en-us/microsoft-365/word}}

\hypertarget{citeproc_bib_item_12}{\textit{Overleaf, Online LaTeX Editor}. (n.d.). Retrieved November 11, 2022, from \url{https://www.overleaf.com}}

\hypertarget{citeproc_bib_item_13}{\textit{Strawberry Perl for Windows}. (n.d.). Retrieved November 28, 2022, from \url{https://strawberryperl.com/}}

\hypertarget{citeproc_bib_item_14}{\textit{Text Editor - Free App for Editing Text Files}. (n.d.). Retrieved November 11, 2022, from \url{https://texteditor.co/}}

\hypertarget{citeproc_bib_item_15}{\textit{Visual Studio Code - Code Editing. Redefined}. (n.d.--a). Retrieved November 28, 2022, from \url{https://code.visualstudio.com/}}

\hypertarget{citeproc_bib_item_16}{\textit{Visual Studio Code - Code Editing. Redefined}. (n.d.--b). Retrieved November 11, 2022, from \url{https://code.visualstudio.com/}}

\hypertarget{citeproc_bib_item_17}{\textit{welcome home : vim online}. (n.d.). Retrieved November 11, 2022, from \url{https://www.vim.org/}}

NO\_ITEM\_DATA:MikTex
\end{hangparas}

\section*{Glossary}
\label{sec:orga11a1cc}
\subsection*{Text Editor}
\label{sec:org5c3d942}
A text editor is a piece of software that facilitates the creation of plain text
files (files that do not contain formatting information). Text editors are most
often used for computer programming, but can be used for any plain text files,
including HTML and \LaTeX{} files.

\subsubsection*{Common Features of a Text Editor}
\label{sec:org7d1fd05}
While the definition of a text editor is simply a piece of software that creates
text files, most text editors also contain extra features that make file
creation easier.

One of the most ubiquitous features is syntax highlighting; this feature changes
the color of specific characters so that the user has an easier time discerning
the purpose of the text.
\begin{figure}[htbp]
\centering
\includegraphics[width=1in]{syntaxhighlighting.png}
\caption{Notice how different pieces of text are highlighted differently? This makes it easier for the document creator to understand.}
\end{figure}

Another feature commonly found in text editors is auto-complete. Auto-complete
will suggest new text based on what is typed. Depending on the implementation,
auto-complete may suggest anything from simple words to complex definitions
written in other files. Some implementations can even utilize artificial
intelligence to make the completions smarter, as is the case with the popular
Github Copilot tool (\citeprocitem{4}{\textit{GitHub Copilot · Your AI Pair Programmer}, n.d.}).

\begin{figure}[htbp]
\centering
\includegraphics[width=1in]{autocomplete.png}
\caption{See how auto-complete suggests a function written earlier in the file? Auto-complete will often change the suggestions based on what has already been written.}
\end{figure}

Text editors also commonly contain an error checkers or \emph{linters}; this is extremely useful in programming related scenarios because some errors can be hard to spot, even to the untrained eye. Linters can also aid in learning, as they often describe the problem in addition to pointing it out. Linters are not limited to just error, they can also show warnings and style recommendations.
\begin{figure}[htbp]
\centering
\includegraphics[width=1in]{linter.png}
\caption{Notice how errors and warnings are differentiated on the left.}
\end{figure}

Some text editors contain many more features than described here,  anything from sending emails to full fledged web browsing.
\subsubsection*{Examples of Text Editors}
\label{sec:org10b77c8}
Many basic text editors exist, the most famous of which is probably Microsoft's Notepad. This text editor does not contain any of the features previously described, aiming to be as bare-bones as possible. On the other end of the spectrum, integrated development environments (IDE) aim to pack as many features as possible regarding a specific programming language or tool. Overleaf (\citeprocitem{12}{\textit{Overleaf, Online LaTeX Editor}, n.d.}) is an online based \LaTeX{} IDE that makes document creation much more intuitive to non-technical audiences.

Some text editors are extensible, meaning that software can be written by third parties to extend, or increase, functionality. Some popular editors in this category are Microsoft's Visual Studio Code (\citeprocitem{16}{\textit{Visual Studio Code - Code Editing. Redefined}, n.d.--b}), GNU Emacs(\citeprocitem{5}{\textit{GNU Emacs - GNU Project}, n.d.}), and Vim (\citeprocitem{17}{\textit{Welcome Home : Vim Online}, n.d.}). These editors have potentially limitless amounts of features and can be modified to fit a variety of different use cases.

Text editors are often built into other software as well. Blender (\citeprocitem{3}{Foundation, n.d.}), contains a editor for use with its built in Python (a programming language) interpreter. Many web based text editors also exist, such as the Text Editor (\citeprocitem{14}{\textit{Text Editor - Free App for Editing Text Files}, n.d.}).

\subsubsection*{Conclusion}
\label{sec:orgeff6c79}
Many different text editors exist and support a variety of different features. Some are very basic, some are extensible, and others are simply pieces of other software. They all share the ability to edit plain text, and most contain at least some quality of life improvements for the user, such as syntax highlighting. They are required to write most programming and markup languages.
\end{document}
